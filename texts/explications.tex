\documentclass[11pt]{article}

\usepackage[utf8]{inputenc}



\begin{document}

\title{Candide 2.0 ou l'optimisme en jeu vidéo }
\author{Nina Ionescu 3mg01}
\date{}
\maketitle

\section{Introduction}
idée de base sur le jeu, brève description
sources : 
Phaser.js par Richard Davey \
Phaser-tilemap-plus par Colin Vella \
Bosca Ceoil par Terry Cavanagh \
Pyxel Edit par Danik ( pseudonyme ) 


\section{Fonctionement du code}
phaser etc.., orienté objet, ECMAScript6  fonctionalité comme les classes etc..
\subsection{Système général}
 choix de la langue 0,1 ; retenir les choix du joueur, les globals etc...
\subsection{Gestion des states du jeu}
qu'est-ce qu'une state, comment ça marche etc...
\subsubsection{Boot et Preload}
boot fait démarrer le jeu, preload pour les assets etc...
\subsubsection{Menu Principal}
expliquer la fonction avec les boutons, dans un premier temps elle est avec une sprite sheet pour chaque langue, éventuelle amélioration.

\subsubsection{Game}
toutes les fonctions agissent ici, dans la partie create de phaser.
\subsubsection{Battle}
expliquer la dynamique de combat, les points de vie etc..
\subsection{Joueur}
expliquer les déplacements, les directions , les interactions etc...
\subsection{Les pnjs}
expliquer la classe, la bulle (classe) 
les interaction avec le joueur etc..
\subsection{Gestion des dialogues}
expliquer la police de caractères, la dynamique des choix, l'objet manager , la syntaxe (les arrays avec les callbacks etc...)
\subsection{Gestion du terrain}
expliquer les maps,collisions, tilesets,calques, warps , l'objet terrainmanager etc... expliquer le fail avec le tween manager pour le fade in et fade out.
\subsection{Interaction avec les objets}
à compléter idée: un objet possède ou non un callback 
\subsection{Gestion du menu et de l'inventaire}
à compléter
\subsection{Du code à l'application}
parler d'éléctron, npm, des packages.json, du favicon, de la fenêtre et des sauvegardes de la partie. (fenêtres)

\subsection{Publication}
système de sauvegarde , gitpages
\section{Scénario alternatif}
expliquer les dialogues, conserver le contenu tout en l'adaptant etc...
\section{Conception des assets}
\subsection{Visuel}
logiciels utilisé , techniques(travailler avec des tiles, \\ couches de transparence...)\\ inspiration pour les personnages, dessins\\ originaux etc.. restrictions au niveau du pixel\\ art, base pour cohérence etc...
\subsubsection{Sprites de dialogues}
techniques(couches, dessins - restrictions), exemple 
\subsubsection{Sprites d'overworld}
techniques, exemples
\subsubsection{Tilesets}
techniques, exemples
\subsection{Musical}
logiciel utilisé , techniques (trouver une suite d'accord, les instruments etc..) , éventuellement les bruitages 
S
\end{document}
