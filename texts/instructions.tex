\documentclass[10pt,a4paper]{article}
\usepackage[utf8]{inputenc}
\usepackage[french]{babel}
\usepackage[T1]{fontenc}
\usepackage{graphicx}
\usepackage{hyperref}
\begin{document}
\begin{center}
\begin{Huge}
Instructions 
\end{Huge}
\end{center}
\vspace{30pt}
\includegraphics{assets/title}\\
\centering
\textit{Note :} Utiliser Google Chrome ou Firefox.\\

\vspace{10pt}
\textbf{Lancer le jeu sur Microsoft Windows }
\begin{itemize}
\item Ouvrir le dossier "webSrc" sur la clé USB.
\item Double-cliquer sur "lanceur-win.exe".
\item Normalement, le jeu démarre sur une page Internet.
\end{itemize}

\vspace{10pt}

\textbf{Lancer le jeu sur MacOS }
\begin{itemize}
\item Ouvrir le dossier "webSrc" sur la clé USB.
\item Double-cliquer sur "lanceur-mac.app".
\item Normalement, le jeu démarre sur une page Internet.
\end{itemize}

\vspace{12pt}
Si le jeu ne s'affiche pas correctement, il est également disponible à l'adresse suivante : \\
\url{oc4.lddr.ch/Ionescu}\\
\vspace{12pt}

Pour créer un serveur local afin de lancer le jeu, le programme \textit{Mongoose }\footnote{\url{https://cesanta.com/binary.html}} de l'entreprise \textit{Cesanta} a été utilisé.



\end{document}